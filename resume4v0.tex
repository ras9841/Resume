\documentclass{article}
% Packages
\usepackage{environ} % custom environments
\usepackage{graphicx} % resize table
\usepackage{enumerate}

% Page Setup
\voffset = -1.25 in
\hoffset = - 1 in
\textwidth = 7 in
\textheight = 10 in
\pagestyle{empty}

% Custom Commands
\setlength\parindent{0cm} % no indent for paragraphs
\newcommand{\fullbar}{\rule{\textwidth}{0.4pt}} % draw line
\newcommand{\heading}[1]{{\Large\textbf{#1}\vspace{-.2cm}\newline\fullbar}} % make header
\renewcommand{\arraystretch}{1} % set table row height
\newcommand{\job}[3]{
	\item\textbf{#1} \hfill #2|#3\\
	}
\newcommand{\proj}[2]{
	\item\textbf{#1} \hfill #2\\
}
\newcommand{\secspace}{.3 cm}
\newcommand{\dash}{\item[-]}

\begin{document}
\heading{Roland Allen Sanford}
LinkedIn: \hspace{.14cm}www.linkedin.com/in/rasanford\hfill	2930 Gaines Basin Rd, Albion, NY 14411\\
GitHub: \quad www.github.com/ras9841 \hfill 585.590.7489 $|$ ras9841@rit.edu
\vspace{.4cm}
	
\heading{Overview}
\begin{flushleft}
	A computational scientist with a background in physics, math, and computer science interested in the construction and analysis of mathematical models of physical systems.
\end{flushleft}  

\vspace{\secspace}
\heading{Educational Highlights}
\begin{flushleft}
	\textbf{Rochester Institute of Technology (RIT)}\hfill Rochester, NY
\end{flushleft}
\textit{Master of Science} \hfill \quad Computational Mathematics \hfill Expected May 2018 \\
\textit{Bachelor of Science} \hfill \ Computational Mathematics \hfill Expected May 2018 \\
\textit{Bachelor of Science} \hfill Physics \hfill Expected May 2018 \\\\
\textbf{Computer Courses} \hfill Parallel and Distributed Systems, Mechanics of Programming, Software Engineering\\
\textbf{Math Courses} \hfill Advanced Linear Algebra, Numerical Analysis, Stochastic Processes, Real Analysis\\
\textbf{Physics Courses} \hfill Classical Mechanics, Electricity and Magnetism, Advanced Laboratory in Physics\\

\vspace{\secspace}
\heading{Relevant Experience}
\begin{description}
\job{Computational Imaging of Atrial Fibrillation}{Feb. 2015}{Present}
\begin{enumerate}
	\vspace{-.6cm}
	\dash Developed the code and documentation for a MATLAB pipeline that reconstructs the epicardial  potentials of a patient's heart by implementing second-order Tikhonov regularization to solve an ill-posed inverse problem. 
	\dash Pipeline includes constructing patient-specific geometric models of the torso and the atria as well as processing electrocardiogram data.
	\dash Delivered phase mapping videos and images resulting from dominant frequency analysis to physician-scientists at Johns Hopkins University to validate our modeling process.
\end{enumerate}

\job{Mathematical Modeling of Contact Lenses}{Jan. 2015}{Present}
\begin{enumerate}
	\vspace{-.6cm}
	\dash Modeled the progression of a soft contact lens and a cylindrically-symmetric eye towards equilibrium.
	\dash Derived the governing PDEs by balancing the stresses in the eye and on its boundaries.
	\dash Discretized the PDEs using forth-order finite difference equations, and solved the resulting coupled system by implementing successive over-relaxation and forward Euler numerical techniques in C++.
\end{enumerate}
\job{Characterization of Heusler Alloys}{Jan. 2014}{Dec. 2014}
\begin{enumerate}
	\vspace{-.6cm}
	\dash Researched the Heusler Alloy $Ni_2Mn(Ga_{1-x}Zn_x$) to determine the effect of Zinc concentration on the alloy's coercivity  and  Curie temperature.
	\dash Prepared,  mounted, and conducted experiments on samples using a Vibrating Sample Magnetometer and an AC Susceptibility rig.
	\dash Wrote MATLAB and python code to aid data acquisition and experimental analysis.
\end{enumerate}
\end{description}
\vspace{\secspace}
\heading{Additional Projects}
\begin{description}
\proj{Chebfun Statistics}{ Ongoing }
	Worked with faculty at the University of Delaware to add a random variable class and associated methods to the Chebfun MATLAB package (www.chebfun.org).
\proj{Contact Lens Web Application}{ Completed }	
	Designed an interactive contact lens application for Alden Optical that displays user-made changes in the lens design in real time (pending release).
\end{description}

\vspace{\secspace}
\heading{Skills}
\textbf{Languages:} \hfill Python, C, Java, MATLAB, \LaTeX; familiar with C++, JavaScript, Julia, and Android\\
\textbf{Operating Systems:} \hfill Windows and Linux (Arch, Fedora, Debian, and Ubuntu); familiar with Mac OS

\end{document}